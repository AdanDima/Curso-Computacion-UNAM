%%%%%%%%%%%%%%%%%%%%%%%%%%%%% Define Article %%%%%%%%%%%%%%%%%%%%%%%%%%%%%%%%%%
\documentclass{article}
%%%%%%%%%%%%%%%%%%%%%%%%%%%%%%%%%%%%%%%%%%%%%%%%%%%%%%%%%%%%%%%%%%%%%%%%%%%%%%%

%%%%%%%%%%%%%%%%%%%%%%%%%%%%% Using Packages %%%%%%%%%%%%%%%%%%%%%%%%%%%%%%%%%%
\usepackage{geometry}
\usepackage[spanish]{babel}
\usepackage{graphicx}
\usepackage{amssymb}
\usepackage{amsmath}
\usepackage{amsthm}
\usepackage{empheq}
\usepackage{mdframed}
\usepackage{booktabs}
\usepackage{lipsum}
\usepackage{graphicx}
\usepackage{color}
\usepackage{psfrag}
\usepackage{pgfplots}
\usepackage{bm}
%%%%%%%%%%%%%%%%%%%%%%%%%%%%%%%%%%%%%%%%%%%%%%%%%%%%%%%%%%%%%%%%%%%%%%%%%%%%%%%

% Other Settings

%%%%%%%%%%%%%%%%%%%%%%%%%% Page Setting %%%%%%%%%%%%%%%%%%%%%%%%%%%%%%%%%%%%%%%
\geometry{a4paper}

%%%%%%%%%%%%%%%%%%%%%%%%%% Define some useful colors %%%%%%%%%%%%%%%%%%%%%%%%%%
\definecolor{ocre}{RGB}{243,102,25}
\definecolor{mygray}{RGB}{243,243,244}
\definecolor{deepGreen}{RGB}{26,111,0}
\definecolor{shallowGreen}{RGB}{235,255,255}
\definecolor{deepBlue}{RGB}{61,124,222}
\definecolor{shallowBlue}{RGB}{235,249,255}
%%%%%%%%%%%%%%%%%%%%%%%%%%%%%%%%%%%%%%%%%%%%%%%%%%%%%%%%%%%%%%%%%%%%%%%%%%%%%%%

%%%%%%%%%%%%%%%%%%%%%%%%%% Define an orangebox command %%%%%%%%%%%%%%%%%%%%%%%%
\newcommand\orangebox[1]{\fcolorbox{ocre}{mygray}{\hspace{1em}#1\hspace{1em}}}
%%%%%%%%%%%%%%%%%%%%%%%%%%%%%%%%%%%%%%%%%%%%%%%%%%%%%%%%%%%%%%%%%%%%%%%%%%%%%%%

%%%%%%%%%%%%%%%%%%%%%%%%%%%% English Environments %%%%%%%%%%%%%%%%%%%%%%%%%%%%%
\newtheoremstyle{mytheoremstyle}{3pt}{3pt}{\normalfont}{0cm}{\rmfamily\bfseries}{}{1em}{{\color{black}\thmname{#1}~\thmnumber{#2}}\thmnote{\,--\,#3}}
\newtheoremstyle{myproblemstyle}{3pt}{3pt}{\normalfont}{0cm}{\rmfamily\bfseries}{}{1em}{{\color{black}\thmname{#1}~\thmnumber{#2}}\thmnote{\,--\,#3}}
\theoremstyle{mytheoremstyle}
\newmdtheoremenv[linewidth=1pt,backgroundcolor=shallowGreen,linecolor=deepGreen,leftmargin=0pt,innerleftmargin=20pt,innerrightmargin=20pt,]{theorem}{Theorem}[section]
\theoremstyle{mytheoremstyle}
\newmdtheoremenv[linewidth=1pt,backgroundcolor=shallowBlue,linecolor=deepBlue,leftmargin=0pt,innerleftmargin=20pt,innerrightmargin=20pt,]{definition}{Definition}[section]
\theoremstyle{myproblemstyle}
\newmdtheoremenv[linecolor=black,leftmargin=0pt,innerleftmargin=10pt,innerrightmargin=10pt,]{problem}{Problem}[section]
%%%%%%%%%%%%%%%%%%%%%%%%%%%%%%%%%%%%%%%%%%%%%%%%%%%%%%%%%%%%%%%%%%%%%%%%%%%%%%%

%%%%%%%%%%%%%%%%%%%%%%%%%%%%%%% Plotting Settings %%%%%%%%%%%%%%%%%%%%%%%%%%%%%
\usepgfplotslibrary{colorbrewer}
\pgfplotsset{width=8cm,compat=1.9}
%%%%%%%%%%%%%%%%%%%%%%%%%%%%%%%%%%%%%%%%%%%%%%%%%%%%%%%%%%%%%%%%%%%%%%%%%%%%%%%

%%%%%%%%%%%%%%%%%%%%%%%%%%%%%%% Title & Author %%%%%%%%%%%%%%%%%%%%%%%%%%%%%%%%
\title{Ejercicio 2 del Tercer Parcial de Calculo Diferencial e Integral 1}
\author{Adan Diaz Martinez \thanks{Profesor Jorge Calderon Espinosa de los Monteron}
\thanks{Emmanuel Ismael González Celio}
\thanks{Alan Hernández Martínez}}
%%%%%%%%%%%%%%%%%%%%%%%%%%%%%%%%%%%%%%%%%%%%%%%%%%%%%%%%%%%%%%%%%%%%%%%%%%%%%%%

\begin{document} 
    \maketitle
    \begin{titlepage}
        \maketitle
    \end{titlepage}
\noindent{\color{red}\Large\textbf{Ejercicio 2.}} Resolver los siguientes ejercicios:

\medskip
\begin{itemize}
    \item Sean $I \subseteq \mathbb{R}$ un intervalo, c $\epsilon$ I un punto y $f:I$  
    $\backslash \{ c\} \to \mathbb{R} $ una función. 
    Demuestre que si $(x_{n})_{n\epsilon \mathbb{N} }$ es una sucesión de puntos en I, todos distintos de c,
    que converge a c y $\lim_{x\to c} f(x)= \textit{L}$, entonces 
    $\lim_{n\to \infty } f(x_{n})= \textit{L}$.


\medskip
$\mathbf{P} \mathbf{D}$
\medskip
$\lim_{n\to \infty}  f(x_{n})= \textit{L}$.


\medskip
{\textcolor{blue}{Prueba}} 
\medskip
Como $\lim_{x\to c} f(x)= \textit{L}$, sabemos que:
\begin{equation}
\forall \epsilon \gneq 0 \medspace \medspace  \exists\delta \gneq 0 : 0\lneq \left\lvert x-c\right\rvert 
\lneq \delta \rightarrow \left\lvert f(x)-\textit{L}\right\rvert \lneq \varepsilon 
\end{equation}
Por otra parte tenemos que $x_{n}\subseteq Domf$ es una sucesion de imagenes que converge a c, es decir:
\begin{equation}
 \lim_{n \to \infty}x_{n}= c \rightarrow \forall n\epsilon \mathbb{N}, \exists N\epsilon \mathbb{N} 
 : n\geq N \rightarrow \left\lvert x_{n}-c\right\rvert \lneq \varepsilon_\bigstar 
\end{equation}
Ademas como $x_{n}\neq c \medspace \medspace \forall n\epsilon \mathbb{N} \rightarrow 0\lneq 
\left\lvert x_{n}-c\right\rvert \lneq \varepsilon_\bigstar $, eso quiere decir que cumple 
con la definicion de limite de función, renombrando $\varepsilon_\bigstar$ como $\delta$ se tiene que: 
\begin{equation}
 \forall n\epsilon\mathbb{N} \medspace  \medspace \exists N\epsilon \mathbb{N} : \medspace si
 \medspace n\geq N \medspace entonces \medspace 0\lneq \left\lvert x_{n}-c\right\rvert \lneq 
 \delta \rightarrow \left\lvert f(x_{n})-\textit{L}\right\rvert \lneq \epsilon  
\end{equation}
\begin{center} 
$\therefore \lim_{n \to \infty } f(x_{n})= \textit{L}$
    \end{center} 
    \begin{flushright} 
        $\blacksquare$
    \end{flushright}
        \item Con base en lo anterior demostrar que $\nexists \lim_{x \to 0} \cos(\frac{1}{x})$ 

        \medskip
        $\mathbf{P} \mathbf{D}$
        \medskip
        $\nexists \lim_{x \to 0} \cos(\frac{1}{x})$
        
        
        \medskip
        {\textcolor{blue}{Prueba}} 
        \medskip
Por la demostracion anterion anterior:
\begin{equation}
 \lim_{x \to 0}\cos(\frac{1}{x})=\textit{L} \medspace si \lim_{n \to \infty}x_{n}=0 \medspace\medspace
x_{n}\neq 0 \medspace \forall n\epsilon \mathbb{N} \rightarrow \lim_{n\to \infty}
    f(x_{n})=\textit{L}
\end{equation}
La prueba es por casos:
\begin{itemize}
    \item {\textcolor{deepGreen}{Caso 1:}} 
\medskip
Sea $(x_{n\diamondsuit})_{n\epsilon \mathbb{N}}=(\frac{1}{2n\pi})\rightarrow 
\lim_{n \to \infty}(\frac{1}{2n\pi})=0$ eso quiere decir:
\begin{equation}
\lim_{n \to \infty}f(x_{n\diamondsuit})=\lim_{n \to \infty}\cos(\frac{1}{\frac{1}{2n\pi}})
=\lim_{n \to \infty}\cos(2n\pi)=1
   \end{equation}
   \item {\textcolor{deepGreen}{Caso 2:}} 
   \medskip
Sea $(x_{n\clubsuit})_{n\epsilon \mathbb{N}}=(\frac{1}{(2n-1)\pi})\rightarrow 
\lim_{n \to \infty}(\frac{1}{(2n-1)\pi})=0$ eso quiere decir:
\begin{equation}
\lim_{n \to \infty}f(x_{n\clubsuit })=\lim_{n \to \infty}\cos(\frac{1}{\frac{1}{(2n-1)\pi}})
=\lim_{n \to \infty}\cos((2n-1)\pi)=-1
   \end{equation}
\end{itemize}
Como los limites de la sucesion de imagenes son distintos:
\begin{equation}
    \rightarrow \nexists \lim_{x \to 0} \cos(\frac{1}{x})
 \end{equation}
 \begin{flushright} 
    $\blacksquare$
\end{flushright}
    \end{itemize}
\end{document}